\chapter{Planteamiento del problema}

\paragraph{Nota:} Presentación basada em el texto: \cite{aalpa}.

\section{Condiciones}

La presente prática gira en torno a la creación de una aplicación en base al
lenguaje de programación funcional ``Caml'' en su versión ligera. Dicha
aplicación ha de ser un juego ``Sudoku Dodeka'', versión de popular juego de
cifras ``Sudoku''. Las reglas de este variante se detallan a continuación:

\begin{itemize}
\item Las posibles fichas del juego contiene las cifras decimales de: 0 a 9 y
las teras A y B.
\item Para todos los elementos de una misma fila no se han de repetir dos fichas
iguales.
\item Para todos los elementos de una misma columna no se han de repetir dos
fichas iguales.
\item Para todos los elementos de una mismo ret\'angulo (3x4) no se han de
repetir dos fichas iguales. 
\item El juego termina cuando se cumplen las reglas anteriormenete expuestas y
no hay ninguna casilla en blanco ('.').
\end{itemize}

\section{Funciones propuestas}

Para el mismo juego se proponen una funciones básicas que deben ser parte del
código. Las misma son:

\begin{itemize}
\item formarSudoku: Toma como argunmento una lista de carácteres (tablero) y
devuelve el Sudoku formateado.
\item mostrar tablero: Recibe un tablero como argumento y devuelve el dibujo del
mismo.
\item elemento casilla tablero: Devuelve el valor que tiene asignado dicha
casilla en el tablero.
\item posibles casilla tablero: Devuelve una lista con los valores que puede
tomar dicha casilla en el tablero.
\item resuelto tablero: Devuelve cierto si el Sudoku está resuelto y falso en
caso contrario.
\end{itemize}

\section{Tipos de Sudoku}

Existe tres tipos de Sudoku planteados:

\begin{itemize}
\item Sudoku fácil: Se trata de un Sudoku sencillo en el que su solución viene
dada por eliminación de posibles valores sobre las distintas casillas.
\item Sudoku difícil: Se trata de un Sudoku en el que la única forma de obtener
su solución se hace con algorítmia compleja (técnica de BackTraking)
\item Sudoku imposible: Se trata de un Sudoku en el que por la disposición de
las fichas en el tablero es imposible llegar a una solución.
\end{itemize}




