\chapter{Pr\'ologo}
He aqu\'i que como nota concluimos. Teor\'ia de Sistemas Operativos Distribuidos
es un libro de texto que como bien su nombre advierte honda y gira en torno a
los fundamentos de estos ``aut\'omatas''. El texto se presenta conciso, se ha
hecho especial incapie en las definiciones, fundamentales para el entendimiento
de la materia. As\'i mismo, se ha utilizado cierta notaci\'on con el objeto de
clarificar mediante s\'imbolos aquello que encierra un significado indivisible.


El texto comienza fundamentando el proceso distribuido. Se trata su
comunicaci\'on y planificaci\'on. Seguido se dan ejemplos de m\'aquinas que
trabajan con este tipo de unidades de datos. 


La segunda parte aborda la gesti\'on de la memoria, tema delicado y principal
para entender que es realmente un sistema distribuido. Por \'ultimo son
analizados los sistemas de ficheros y tres casos implementados: \textit{MS-DOS}
(aun no siendo puramente distribuido), \textit{NFTS} y el sistema de ficheros de
UNIX SYSTEM V, \textit{UFS}.


Espero que el trabajo vertido en est\'a obra sea de su agrado.


Reciba un cordial saludo. Diego Antonio Lucena Pumar.