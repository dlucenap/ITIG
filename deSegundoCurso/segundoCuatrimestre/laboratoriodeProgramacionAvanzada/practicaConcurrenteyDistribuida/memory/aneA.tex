\chapter{El sistema de Tiempo Aleatorio}

En este anexo haremos un inciso sobre el sistema que empleamos para determinar
un tiempo aleatorio en Java.

Cuando es necesario que un determinado Hilo espere un tiempo
invocamos al m\'etdo \textit{sleep(l\'imite de tiempo)}.

En est\'a pr\'actica hemos utilizado un m\'etdo alatorio de \textit{random()}
para la clase \textit{Math}. La ecuaci\'on como se ver\'a a continuaci\'on, sirve de ejemplo para determinar
un tiempo aleatorio entre 800 y 1500 milisegundos:

\begin{equation}
RANDOM * (LIMITE - BASE) + BASE
\end{equation}

Por lo que si \textit{random()} es 0 dar\'a la base y siendo uno ser\'a: 800 + 700 = 1500,
que es el tiempo l\'imite m\'aximo.

Ahora mostramos un ejemplo de la utilidad:

\begin{verbatim}
try {
    sleep(((long) (Math.random() * (1500 - 800)) + 800));
    } catch (InterruptedException e) {
	System.out.println(e.getMessage());
				      }		
\end{verbatim}

